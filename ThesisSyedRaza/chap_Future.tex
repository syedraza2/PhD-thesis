\chapter{Future Directions}\label{chap:Future}

Dirac/Weyl (semi)metals are a specific type of nodal electronic matter. For example, nodal superconductors were studied in states with dx$^2$-y$^2$ pairing~\cite{RyuHatsugaiPRL02}, He$^3$ in its superfluid A-phase~\cite{Volovik3HeA,Volovikbook}, and non-centrosymmetric states~\cite{SchnyderRyuFlat,BrydonSchnyderTimmFlat}. Weyl and Dirac fermions were generalized in \TR and mirror symmetric systems to carry $\mathbb{Z}_2$ topological charge~\cite{morimotoFurusakiPRB14}. General classification and characterization of gapless nodal semimetals and superconductors were proposed~\cite{Sato_Crystalline_PRB14,ZhaoWangPRL13,ZhaoWangPRB14,ChiuSchnyder14,matsuuraNJP13,Volovikbook,RMP,HoravaPRL05}. It would be interesting to investigate the effect of strong many-body interactions in general nodal systems.

%coarse-graining implication in real space RG and interaction; vortex dynamics
In Sec.~\ref{sec:DiracSemimetal}, we described a coarse-graining procedure of the coupled wire model that resembles a real-space renormalization and allows one to integrate out high energy degrees of freedom. While this procedure was not required in the discussions that follow because the many-body interacting model we considered was exactly solvable, it may be useful in the analysis of generic interactions and disorder. The coarse-graining procedure relied on the formation of vortices, which were introduced extrinsically. Like superconducting vortices, it would be interesting as a theory and essential in application to study the mechanism where the vortices of Dirac mass can be generated dynamically. To this end, it may be helpful to explore the interplay between possible (anti)ferromagnetic orders and the spin-momentum locked Dirac fermion through antisymmetric exchange interactions like the Dzyaloshinskii-Moriya interaction~\cite{Dzyaloshinsky58,Moriya60}.

%topological order, threefold lattice and alternative fractionalization
The symmetry-preserving many-body gapping interactions considered in Sec.~\ref{sec:intenable} have a ground state that exhibits long-range entanglement. This entails degenerate ground states when the system is compactified on a closed three dimensional manifold, and fractional quasi-particle and quasi-string excitations or defects. These topological order properties were not elaborated in our current work but will be crucial in understanding the topological phase~\cite{SirotaRazaTeoappearsoon} as well as the future designs of detection and observation. It would also be interesting to explore possible relationships between the coupled wire construction and alternative exotic states in three dimensions, such as the Haah's code~\cite{Haah11,Haah13}.

The many-body inter-wire backscatterings proposed in Sec.~\ref{sec:interactionmodels} were based on a fractionalization scheme described in \ref{sec:gluing} that decomposes a chiral Dirac channel with $(c,\nu)=(1,1)$ into a decoupled pair of Pfaffian ones each with $(c,\nu)=(1/2,1/2)$. In theory, there are more exotic alternative partitions. For instance, if a Dirac channel can be split into three equal parts instead of two, an alternative coupled wire model that put Dirac channels on a honeycomb vortex lattice could be constructed by backscattering these fractionalized channels between adjacent pairs of wires. Such higher order decompositions may already be available as conformal embeddings in the \CFT context. For example, the affine $SU(2)$ Kac-Moody theory at level $k=16$ has the central charge $c=8/3$, and its variation may serve as the basis of a ``ternionic" model.

In this work, we considered two models but the procedure and theoretical framework can be extended to a number of other interacting three-dimensional models with different sets of symmetries. We expect them to give a whole range of new three-dimensional topological orders. It would be interesting to have a general classification procedure of these SET states, but it remains unclear for now how to combine crystal symmetries and conformal field theories in this description. 

In the current models, the many-body interaction is between wires in a planar direction which effectively leads to stacked gapped layers of topological order coupled together. It would be interesting to see if the many-body interactions can be introduced in both planar and inter-layer directions to get a topologically ordered phase.  

We discussed the fractional excitations as part of the topological order that can arise in these gapped states. However, work on the complete characterization and braiding statistics of these excitations is in progress and will appear soon. One of the goals is to build a non-Abelian three-dimensional topological order beyond what is presented in \cite{Iadecola2017}. We believe this can be built out of the N=odd case in the gapped interacting Dirac nodal superconductor.

Recently, there has been work on topological phase transitions between different topologically ordered states in the Kitaev model \cite{ZouHe2018}. One possible direction is to study if there can be phase transitions between the various topologically ordered phases of the Dirac nodal superconductor. 

The current work relies on a coupled-wire description of obtaining three-dimensional topological order. A future direction would be to come up with a fermion-fermion interaction description to realize 3D topological order, which may be more useful for realizing materials. We gave some antiferromagnetic stabilization arguments for the topologically ordered phases in the gapped states. In the future, a more rigorous stability analysis can be done using traditional RG flow methods. 

Another avenue to explore is extending the coupled wire description to time-dependent Floquet systems. Although non-interacting topological phases have been well-studied, interacting Floquet systems are still an area of intense interest and coupled-wire models might be useful in them.  
