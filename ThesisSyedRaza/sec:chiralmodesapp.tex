\chapter{Chiral modes along topological defects}\label{sec:chiralmodesapp}
In section~\ref{sec:DiracSemimetal}, we begin with the Dirac Hamiltonian \eqref{DiracHam} where the mass term winds around a vortex and as a consequence, it hosts a chiral Dirac channel along the vortex (also see figure~\ref{fig:Diracstring}). Here we will demonstrate an example of a simple vortex, and show that there is a chiral Dirac zero mode. In general, the correspondence between the number of protected chiral Dirac channels and the vortex winding is a special case of the Atiyah-Singer Index theorem~\cite{AtiyahSinger63} and falls in the physical classification of topological defects~\cite{TeoKane}.

First, say we start with the Hamiltonian from \eqref{DiracHam}. Then for simplicity we consider the particular Dirac mass $m({\bf r})=m_x({\bf r})+im_y({\bf r})=|m|e^{i\theta}$ that constitute a vortex along the $z$-axis, where $\theta$ is the polar angle on the $xy$-plane. By replacing $k_{x,y}\leftrightarrow-i\partial_{x,y}$, \eqref{DiracHam} becomes \begin{align}H({\bf r})=&\hbar v(-i\partial_xs_x-i\partial_ys_y+k_zs_z)\mu_z\nonumber\\&\;+|m|\cos\theta\mu_x+|m|\sin\theta\mu_y\label{DiracHamapp}\end{align} where $k_z$ is still a good quantum number because translation in $z$ is still preserved. The Hamiltonian can be transformed under a new basis into \begin{align}H'=UHU^{-1}=\left(\begin{smallmatrix}-\hbar vk_z&D\\D^\dagger&\hbar vk_z\end{smallmatrix}\right),\quad U =\left(\begin{smallmatrix}0&1&0&0\\0&0&1&0\\1&0&0&0\\0&0&0&1\end{smallmatrix}\right)\end{align} where the Dirac operator occupying the off-diagonal blocks is \begin{align}D^\dagger &=\left(\begin{smallmatrix}-2i\hbar v\partial_w&|m|e^{-i\theta}\\|m|e^{i\theta}&2i\hbar v \partial_{\bar{w}}\end{smallmatrix}\right)\nonumber\\&=e^{-i\theta\sigma_z}\left(\begin{smallmatrix}-i\hbar v(\partial_r-i \partial_\theta/r)&|m|\\|m|&i\hbar v(\partial_r+i\partial_\theta/r)\end{smallmatrix}\right)\end{align} where $w=x+iy=re^{i\theta}$ and $\sigma_z=\mathrm{diag}(1,-1)$. 

Now we separate the Hamiltonian \begin{align}H'(k_z)=\hbar vk_z\Gamma_5+\left(\begin{smallmatrix}0&D\\D^\dagger&0\end{smallmatrix}\right).\end{align} where $\Gamma_5=\mathrm{diag}(-\openone_2,\openone_2)$. We note that the zero momentum sector $H'(k_z=0)$ has a chiral symmetry since it anticommutes with with $\Gamma_5$, and it reduces to the Jackiw-Rossi vortex problem in two-dimensions~\cite{JackiwRossi81}. The Dirac operator $D^\dagger$ has only one normalizable zero mode $u_0(r)\propto e^{-|m|r/\hbar v}(e^{i\pi/4}, e^{-i\pi/4})^T$, while its conjugate $D$ has none. $H'(k_z=0)$ therefore has a zero eigenvector of $\psi_0(r)=(u_0(r),0)^T$, which is also an eigenvector of $\Gamma_5$. In the full Hamiltonian, the zero mode $\psi_0(r)$ has energy $-\hbar vk_z$ and corresponds a single mid-gap chiral Dirac channel.