\chapter{Chiral modes along topological defects}\label{sec:chiralmodesapp}
In Sec.~\ref{sec:DiracSemimetal}, we begin with the Dirac Hamiltonian \eqref{DiracHam} where the mass term winds around a vortex and as a consequence, it hosts a chiral Dirac channel along the vortex (also see Fig.~\ref{fig:Diracstring}). Here we will demonstrate an example of a simple vortex, and show that there is a chiral Dirac zero mode. In general, the correspondence between the number of protected chiral Dirac channels and the vortex winding is a special case of the Atiyah-Singer Index theorem~\cite{AtiyahSinger63} and falls in the physical classification of topological defects~\cite{TeoKane}.

First, say we start with the Hamiltonian from \eqref{DiracHam}. Then for simplicity we consider the particular Dirac mass $m({\bf r})=m_x({\bf r})+im_y({\bf r})=|m|e^{i\theta}$ that constitute a vortex along the $z$-axis, where $\theta$ is the polar angle on the $xy$-plane. By replacing $k_{x,y}\leftrightarrow-i\partial_{x,y}$, \eqref{DiracHam} becomes \begin{align}H({\bf r})=&\hbar v(-i\partial_xs_x-i\partial_ys_y+k_zs_z)\mu_z\nonumber\\&\;+|m|\cos\theta\mu_x+|m|\sin\theta\mu_y\label{DiracHamapp}\end{align} where $k_z$ is still a good quantum number because translation in $z$ is still preserved. The Hamiltonian can be transformed under a new basis into \begin{align}H'=UHU^{-1}=\left(\begin{smallmatrix}-\hbar vk_z&D\\D^\dagger&\hbar vk_z\end{smallmatrix}\right),\quad U =\left(\begin{smallmatrix}0&1&0&0\\0&0&1&0\\1&0&0&0\\0&0&0&1\end{smallmatrix}\right)\end{align} where the Dirac operator occupying the off-diagonal blocks is \begin{align}D^\dagger &=\left(\begin{smallmatrix}-2i\hbar v\partial_w&|m|e^{-i\theta}\\|m|e^{i\theta}&2i\hbar v \partial_{\bar{w}}\end{smallmatrix}\right)\nonumber\\&=e^{-i\theta\sigma_z}\left(\begin{smallmatrix}-i\hbar v(\partial_r-i \partial_\theta/r)&|m|\\|m|&i\hbar v(\partial_r+i\partial_\theta/r)\end{smallmatrix}\right)\end{align} where $w=x+iy=re^{i\theta}$ and $\sigma_z=\mathrm{diag}(1,-1)$. 

Now we separate the Hamiltonian \begin{align}H'(k_z)=\hbar vk_z\Gamma_5+\left(\begin{smallmatrix}0&D\\D^\dagger&0\end{smallmatrix}\right).\end{align} where $\Gamma_5=\mathrm{diag}(-1_2,1_2)$. We note that the zero momentum sector $H'(k_z=0)$ has a chiral symmetry since it anticommutes with with $\Gamma_5$, and it reduces to the Jackiw-Rossi vortex problem in two-dimensions~\cite{JackiwRossi81}. The Dirac operator $D^\dagger$ has only one normalizable zero mode $u_0(r)\propto e^{-|m|r/\hbar v}(e^{i\pi/4}, e^{-i\pi/4})^T$, while its conjugate $D$ has none. $H'(k_z=0)$ therefore has a zero eigenvector of $\psi_0(r)=(u_0(r),0)^T$, which is also an eigenvector of $\Gamma_5$. In the full Hamiltonian, the zero mode $\psi_0(r)$ has energy $-\hbar vk_z$ and corresponds a single mid-gap chiral Dirac channel.

\chapter{Symmetry transformations of Chern invariants}\label{sec:Chernapp}

In Sec.~\ref{sec:anomaly}, we discussed the Chern numbers on two-dimensional momentum planes of the anomalous Dirac (semi)metal. It was claimed that the Chern numbers \eqref{1stChern} on the two planes at $k_x=\pm\pi/2$ (see Fig.~\ref{fig:Weylspectrum}) are of opposite signs because of the \AFTR and twofold $\mathcal{C}_2$ (screw) rotation symmetries. In this appendix we will derive the symmetry flipping operations on the Chern invariants.

We begin with a Bloch Hamiltonian $H({\bf k})$ that is symmetric under the operation $G({\bf k})$, \begin{align}H({\bf k})&=G(g{\bf k})H(g{\bf k})G(g{\bf k})^{-1}\end{align} if $G$ is unitary, or \begin{align}H({\bf k})&=G(g{\bf k})H(g{\bf k})^\ast G(g{\bf k})^{-1}\end{align} if it is antiunitary. Let $|u_m({\bf k})\rangle$ be the occupied states of $H({\bf k})$. We define $|u'_m({\bf k})\rangle=|Gu_m({\bf k})\rangle=G(g{\bf k})|u_m(g{\bf k})\rangle$ (or $|u'_m({\bf k})\rangle=|Gu_m({\bf k})\rangle=G(g{\bf k})|u_m(g{\bf k})^\ast\rangle$), which is also an occupied state of $H({\bf k})$, for unitary (resp.~antiunitary) symmetry.

The Chern number \eqref{1stChern} can equivalently be defined as \begin{align}\mathrm{Ch}_1(k_x)=\frac{i}{2\pi}\int_{\mathcal{N}_{k_x}}\mathrm{Tr}\left(\mathcal{F}_{\bf k}\right)\label{1stChernapp}\end{align} where $\mathrm{Tr}\left(\mathcal{F}_{\bf k}\right)=d\mathrm{Tr}\left(\mathcal{A}_k\right)$, $\mathcal{N}_{k_x}$ is the oriented $k_yk_z$-plane with fixed $k_x$, and $\mathcal{A}_k$ is the Berry connection of the occupied states $\mathcal{A}_{\bf k}^{mn}=\langle u_m({\bf k})|du_n({\bf k})\rangle$. The Berry connection transforms according to \begin{align}{\mathcal{A}'}_{\bf k}^{mn}&\equiv\langle u'_m({\bf k})|du'_n({\bf k})\rangle\\&=\langle u_m(g{\bf k})|G(g{\bf k})^\dagger d \left[G(g{\bf k})|u_n(g{\bf k})\rangle\right]\nonumber\\&=\mathcal{A}_{g{\bf k}}^{mn}+\langle u_m(g{\bf k})|\left[G(g {\bf k})^\dagger dG(g {\bf k})\right]|u_n(g{\bf k})\rangle\nonumber\end{align} for unitary $G$, or \begin{align}{\mathcal{A}'}_{\bf k}^{mn}&=\left(\mathcal{A}_{g{\bf k}}^{mn}\right)^\ast+\langle u_m(g{\bf k})^\ast|\left[G(g {\bf k})^\dagger dG(g {\bf k})\right]|u_n(g{\bf k})^\ast\rangle\nonumber\\&=-\mathcal{A}_{g{\bf k}}^{nm}+\langle u_m(g{\bf k})^\ast|\left[G(g {\bf k})^\dagger dG(g {\bf k})\right]|u_n(g{\bf k})^\ast\rangle\nonumber\end{align} if $G$ is antiunitary, because the connection is skew-hermitian $\mathcal{A}=-\mathcal{A}^\dagger$. Therefore \begin{align}%\mathrm{Tr}(\mathcal{A}'_{\bf k})&=\mathrm{Tr}(\mathcal{A}_{g{\bf k}})+\mathrm{Tr}\left\{P_{g \bf{k}}\wedge\left[G(g{\bf k})^\dagger dG(g {\bf k})\right]\right\},\nonumber\\
\mathcal{F}'_{\bf k}&=\mathcal{F}_{g{\bf k}}+d\mathrm{Tr}\left\{P_{g\bf{k}}\wedge\left(G(g{\bf k})^\dagger dG(g{\bf k})\right]\right\}\label{curvature}\end{align} for an unitary symmetry, or \begin{align}%\mathrm{Tr}(\mathcal{A}'_{\bf k})&=-\mathrm{Tr}(\mathcal{A}_{g{\bf k}})+\mathrm{Tr}\left\{P_{g \bf{k}}^\ast\wedge\left[G(g{\bf k})^\dagger dG(g {\bf k})\right]\right\},\nonumber\\
\mathcal{F}'_{\bf k}&=-\mathcal{F}_{g{\bf k}}+d\mathrm{Tr}\left\{P_{g\bf{k}}^\ast\wedge\left(G(g{\bf k})^\dagger dG(g{\bf k})\right]\right\}\label{curvature2}\end{align} for an antiunitary one. Here $P({\bf k})=\sum_n|u_n({\bf k})\rangle\langle u_n({\bf k})|$ is the projection operator on to the occupied energy states at momentum ${\bf k}$. Since the trace of Berry curvature $\mathrm{Tr}(\mathcal{F})$ does not depend on the gauge choice of occupied states, $\mathrm{Tr}(\mathcal{F}_{\bf k})=\mathrm{Tr}(\mathcal{F}'_{\bf k})$. We notice the final terms in both \eqref{curvature} and \eqref{curvature2} integrate to zero over the closed periodic momentum plane $\mathcal{N}_{k_x}$. This is because they are total derivatives, and unlike $\mathcal{A}_{\bf k}$, $P_{\bf k}$ and $G({\bf k})$ are defined non-singularly on the entire Brillouin zone (see \eqref{AFTRk} and \eqref{C2k}). 

Now we derive the relation between the Chern number \eqref{1stChernapp} between $k_x$ and $-k_x$ using the antiunitary \AFTR and the unitary $\mathcal{C}_2$ symmetries. The \AFTR symmetries flip all momentum axes $\mathcal{T}_{11},\mathcal{T}_{\bar{1}1}:(k_x,k_y,k_z)\mapsto(-k_x,-k_y,-k_z)$, while the $\mathcal{C}_2$ symmetry flips only two $\mathcal{C}_2:(k_x,k_y,k_z)\mapsto(-k_x,-k_y,k_z)$. Thus, $\mathcal{T}_{11},\mathcal{T}_{\bar{1}1}:\mathcal{N}_{k_x}\to\mathcal{N}_{-k_x}$ maps between opposite planes while preserving their orientations, but $\mathcal{C}_2:\mathcal{N}_{k_x}\to-\mathcal{N}_{-k_x}$ is orientation reversing. Lastly, we substitute \eqref{curvature} and \eqref{curvature2} into \eqref{1stChernapp}, and apply a change of integration variable ${\bf k}\leftrightarrow g{\bf k}$. The \AFTR and $\mathcal{C}_2$ requires the Chern number to flip under $k_x\leftrightarrow-k_x$ \begin{align}\mathrm{Ch}_1(k_x)=-\mathrm{Ch}_1(-k_x).\end{align}


