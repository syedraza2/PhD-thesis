\chapter{Conclusion}\label{chap:Conclusion}

Dirac and Weyl (semi)metals have generated immense theoretical and experimental interest. On the experimental front, this is fueled by an abundant variety of material classes and their detectable \ARPES and transport signatures. On the theoretical front, Dirac/Weyl (semi)metal is the parent state that, under appropriate perturbations, can give birth to a wide range of topological phases, such as topological (crystalline) insulators and superconductors. In this work, we explored the consequences of a specific type of strong many-body interaction based on a coupled-wire description. In particular, we showed that (i) a 3D Dirac fermion can acquire a finite excitation energy gap in the many-body setting while preserving the symmetries that forbid a single-body Dirac mass, and (ii) interaction can enable an anomalous antiferromagnetic time-reversal symmetric topological (semi)metal whose low-energy gapless degrees of freedom are entirely described by a pair of non-interacting electronic Weyl nodes separated in momentum space. We also extended this model to the superconducting analogs of the Dirac Weyl semimetals. In \ref{chap:Model3}, we have explicitly constructed various forms of many body interactions that open gaps in the energy spectrum while preserving the underlying symmetries present in coupled wire constructions of $3D$ Dirac nodal superconductors. In Sec.~\ref{sec:manybody1}, we found that the gapped bulk of the two-dimensional $y-z$ plane supports non-local fractional quasi-particles that develop the non-trivial topological orders. When the system is extended into the full three-dimensions, the fractional excitations can be still maintained to generate the topological degeneracy. In this work, we indicate that the many-body interactions generate non-trivial topological orders in three dimensions. However, it still remains unresolved that how these non-local excitations behave in three dimensions. The detailed physical behavior of these fractional particles can be studied in future works.

Furthermore, we constructed a unimodular $E_8$ gapping potential when there are $N=16$ Dirac channels along a vortex line. To build the $E_8$ gapping potential, we utilized $SO(32)\sim E_8 \times E_8$ decomposition. The resulting gapped phase did not support the topological order due to the unimodular property of the $E_8$ lattice. In general, even unimodular lattices exist in every dimensions multiples of $8$. 


A brief conceptual summary was presented in Sec.~\ref{chap:Summary}. We include a short discussion on what are the broad implications of this work and then discuss possible future directions:

\textbf{Theoretical impact:} We believe this work is a first step towards a duality between quantum critical transitions of short-range entangled symmetry-protected topological phases and long-range entangled symmetry-enriched topological phases. The 3D topological order in these phases is completely different from 2D topological order as it can have both point and line-like excitations and a much richer structure ~\cite{SirotaRazaTeoappearsoon}. There have been field-theoretical descriptions, along the lines of BF and Chern-Simons theories, of 3D topological phases that support these richer structures, such as loop braiding. However, there have been only very few exact solvable examples and none of them patch the field-theoretical descriptions and microscopic electronic systems. The construction presented in this work opens a new direction towards making such a connection. The models are exactly solvable, and they originate from a microscopic Dirac electronic system with local 2-body interactions. They also have potential impact on numerical modelling. For example, the interacting coupled wire model can be approached by a lattice electronic model, which forgoes exact solvability but potentially leads to new critical transitions between topological phases in 3D.

\textbf{High-energy impact:} For a single pair of Weyl nodes with opposite chirality, time-reversal symmetry (TRS) must be broken. Hence, for time-reversal symmetric systems, at least 4 Weyl nodes are required. In this dissertation, we have shown that, as enabled by many-body interactions, an electronic system can support a single pair of Weyl nodes in low-energy without violating TRS (c.f.~Subsec.~\ref{sec:intenable}). Such a material, if it exists, can be verified experimentally by \ARPES, and as a non-trivial consequence, our results assert that such a material must encode long-range entanglement. The existence of a single pair of massless Weyl fermions without TRS breaking in 3+1D can potentially provide new theories beyond the standard model.

\textbf{Experimental realization:} There have been numerous field-theoretical discussions on possible properties of topologically ordered phases in 3D ~\cite{BiYouXu14,JiangMesarosRan14,WangLevin14,JianQi14,WangWen15,WangLevin15,LinLevin15,ChenTiwariRyu15}. However, unlike the 2D case there are no materials that exhibit topological order (quasiparticle excitations) in 3D. In this work, we show that an interacting Weyl or Dirac semimetal is a good place to start for the following reasons. A symmetry preserving gap must result in topological order and fractionalization (c.f.~Subsec.~\ref{sec:intenable}). While it is entirely likely that interactions leads to a spontaneous symmetry breaking phase, we show that there is no obstruction to realizing an interacting phase that preserves symmetries. Such a gapping must support fractionalization, such as the $e/4$ charged Ising-like and $e/2$ charged semion-like quasiparticles in the bulk, as predicted by our work. These charged particles can in principle be measured using a shot noise experiment across a point contact. Moreover, the gapping procedure involves Pfaffian channels so there should be excitations that mirror those in a Pfaffian state. There would also be line-like excitations in 3D for which the experimental signature is not yet clear. Therefore, we believe an interaction-enabled, symmetry-preserving gapped Weyl/Dirac semimetal is a good candidate for realizing topologically ordered phases in 3D. As for the experimental verification of the anomalous interaction-enabled Dirac semimetal, the electronic energy spectrum of a single pair of momentum-separated Weyl nodes in the presence of time-reversal symmetry can be measured using \hyperlink{ARPES}{ARPES}, assuming spontaneous symmetry-breaking is absent. Although the proposed experimental signatures, if measured, will strongly point towards the existence of such states, we cannot claim that such signatures provide a smoking gun evidence yet. More work needs to be done for the complete characterization of the point-like and line-like topological order of these states and will be part of a future work.

Apart from the 3D topological order having a much richer structure than 2D topological order, the 3D case presented in this work is qualitatively different from the well-studied 2D case. In the 2D case, the massless Dirac surface state is anomalous and lives on the boundary of a higher dimensional bulk. This is qualitatively distinct from the 3D Dirac/Weyl (semi)metal, which does not require holographic projection from a 4D bulk. In fact, a single 3D Weyl fermion, which is supported on the boundary of a 4D topological insulator, cannot be gapped while preserving charge U(1) conservation even with many-body interaction due to chiral anomaly. This serves as a counter-example which distinguishes the gappability of 2D versus 3D boundary state. Thus, it is not a priori an expected result that a Dirac/Weyl (semi)metal can be gapped without breaking symmetries. Moreover, the topological origin of 3D Dirac/Weyl (semi)metals relies on the addition of non-local spatial symmetry, in the current 3D case, the $C_2$ screw rotation. This is distinct from the 2D Dirac surface case, where all symmetries are local.


